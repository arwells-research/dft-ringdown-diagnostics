\section{Demonstration on synthetic NR-like ringdowns}
\label{sec:toy_demo}

We demonstrate the proposed diagnostics using synthetic ringdown data designed to mimic
numerical-relativity (NR) ringdown structure while isolating operator-theoretic effects from
detector noise and numerical artifacts.

\subsection{Signal model}

We consider a synthetic strain-like observable
\begin{equation}
  h(t) = \Re\!\left[
    \sum_{n=0}^{N} A_n\,
    e^{(-\alpha_n + i\omega_n)(t-t_0)} e^{i\phi_n}
  \right]
  + h_{\mathrm{prompt}}(t),
  \label{eq:toy_signal}
\end{equation}
where $n=0$ corresponds to the fundamental mode (e.g.\ ``220''), $n=1$ to the first overtone
(e.g.\ ``221''), and $n=2$ to the second overtone (e.g.\ ``222''). The term
$h_{\mathrm{prompt}}(t)$ models a short-lived non-modal transient associated with merger
dynamics.

\emph{Prompt term specification.}
For reproducibility we take
\begin{equation}
  h_{\mathrm{prompt}}(t) = A_p \exp\!\left(-\frac{(t-t_0)^2}{2\sigma^2}\right)
  \cos\!\big(\omega_p (t-t_0) + \phi_p\big),
  \label{eq:prompt_term}
\end{equation}
with $\sigma/M=1$, $\omega_p M=0.6$, and $\phi_p$ drawn uniformly on $[0,2\pi)$; qualitative
conclusions are insensitive to moderate variations of these parameters.

\emph{Representative parameters.}
Unless otherwise stated, we set units by the remnant mass $M$ and use values chosen to be
broadly Kerr-like in overtone hierarchy:
\begin{align}
  \omega_0 M &= 0.53, & \alpha_0 M &= 0.08, & A_0 &= 1.0, \nonumber\\
  \omega_1 M &= 0.50, & \alpha_1 M &= 0.24, & A_1 &= 0.6, \nonumber\\
  \omega_2 M &= 0.47, & \alpha_2 M &= 0.40, & A_2 &= 0.3,
  \label{eq:toy_params}
\end{align}
with phases $\phi_n$ drawn uniformly on $[0,2\pi)$.
The prompt term $h_{\mathrm{prompt}}(t)$ is taken to be a rapidly decaying
Gaussian-modulated oscillation with support primarily for $(t-t_0)/M \lesssim 5$.

\subsection{Reconstructions}

We compare truncated reconstructions using:
(i) the fundamental only,
(ii) fundamental plus first overtone,
(iii) fundamental plus first two overtones,
mirroring strategies commonly used in NR-calibrated ringdown analyses
\cite{Giesler2019,Isi2019}. Each reconstruction is fit over a family of start times $t_0$ using
least-squares minimization of waveform residuals.

\subsection{Closure-energy diagnostics}

For each reconstruction we compute the closure energy $\Eop(\hat{x}(t))$ defined in
\cref{eq:closure_energy}, using a fixed canonical weight $W$ and a linearized closure residual
operator $D\Rop_{X_\star}$ chosen to penalize projection inconsistency between the radiative
and asymptotically stationary representations.

We verified in synthetic tests that replacing $W$ by nearby physically motivated alternatives
(e.g.\ equivalent norms under residual-space reparameterizations or mild curvature-weighted
variants) preserves the qualitative monotonicity classification and inferred $t_\star$; a
systematic catalog-scale sensitivity study is deferred to future work.

A particularly transparent diagnostic is a plot of closure energy versus time for competing
reconstructions:
\begin{equation}
  \Eop_{(0)}(t), \quad \Eop_{(0,1)}(t), \quad \Eop_{(0,1,2)}(t).
\end{equation}
Figure~\ref{fig:closure_energy} shows a representative example.

\begin{figure}[t]
  \centering
  \includegraphics[width=\linewidth]{closure_energy_vs_time.pdf}
  \caption{
  Closure energy $\mathcal{E}(\hat{x}(t))$ as a function of time for competing truncated
  modal reconstructions of a synthetic NR-like ringdown signal. Curves correspond to
  reconstructions using the fundamental mode only (0), the fundamental plus first overtone
  (0,1), and the fundamental plus first two overtones (0,1,2). At early times
  $(t-t_0)/M \lesssim t_\star/M \sim 8$, inclusion of additional overtones reduces waveform
  residuals but induces oscillations or transient growth in $\mathcal{E}$, violating the
  monotonicity condition of the legitimacy criterion. This behavior indicates
  pseudospectral overlap and non-normal mixing rather than independent relaxation channels.
  For $t \ge t_\star$, closure energy decays monotonically and reconstructions stabilize
  under window shifts, marking the onset of physically interpretable ringdown.
  }
  \label{fig:closure_energy}
\end{figure}

At early times $(t-t_0)/M \lesssim t_\star/M \sim 8$, inclusion of additional overtones reduces
waveform residuals but induces oscillations or transient growth in $\Eop(\hat{x}(t))$, violating
the legitimacy criterion \cref{eq:legitimacy}. In particular, the $(0,1,2)$ reconstruction
exhibits clear intervals with $d\Eop/dt>0$, despite achieving the lowest instantaneous
waveform mismatch.

For $t \ge t_\star$, pseudospectral overlap diminishes, resolvent isolation improves, and the
$(0)$ and $(0,1)$ reconstructions exhibit monotone closure-energy decay. In this regime, modal
coefficients stabilize under window shifts and basis refinements, indicating entry into a
physically interpretable ringdown phase.

While non-normality provides a natural explanation for early-time mode ambiguity, nonlinear
effects and higher-order perturbations may also contribute during the prompt response; the
present framework does not model these effects explicitly, but instead supplies a criterion
for when linear modal interpretations become reliable.

\subsection{Interpretation of pseudospectral artifacts}

Classifying an early-time component as a ``pseudospectral artifact'' does not imply that the
corresponding Kerr QNM is unphysical or absent from the spectrum
\cite{Ota2021,Boyd2022}. Rather, it indicates that the mode is not yet spectrally isolated from
the surrounding pseudospectrum, so its coefficient cannot be interpreted as an independent
relaxation amplitude.

In this regime, additional basis elements may be mathematically required to reconstruct the
signal, but they function as non-modal patches compensating for non-normal mixing rather than
as asymptotic decay channels. Only once resolvent isolation and monotone closure-energy decay
are achieved do overtone amplitudes acquire physical meaning as independent relaxation modes.

\paragraph*{Noise robustness.}
The diagnostics introduced here are intended primarily for high–signal-to-noise ratio (SNR)
events, where early-time modeling ambiguities dominate statistical uncertainty. In the
presence of realistic detector noise, instantaneous estimates of $d\Eop/dt$ may exhibit
short-time fluctuations; accordingly, the legitimacy criterion is applied to time-averaged or
smoothed closure-energy trends rather than pointwise derivatives. 
In the results shown here, smoothing is performed over windows of duration $\Delta t/M \sim 1\text{--}3$,
and conclusions are stable under modest variations in this range.
Synthetic injections with
stationary Gaussian noise indicate that the monotonicity classification is stable provided the
signal-to-noise ratio in the dominant mode exceeds $\mathcal{O}(20)$ over the analysis window;
in tests at representative values $\mathrm{SNR}=15,\,30,$ and $50$, the diagnostic degrades
gracefully at low SNR by deferring $t_\star$ to later times rather than producing
false-positive mode identifications. A systematic study across LIGO–Virgo–KAGRA noise
realizations is deferred to future work.
