% ============================
% Appendix A: Example realization of the closure residual
% ============================
\appendix
\section{Example realization of the closure residual in Kerr ringdown}
\label{app:closure_residual_example}

This appendix provides a concrete (illustrative) realization of the abstract closure residual
operator $\Rop$ introduced in \cref{sec:closure_legitimacy}. The purpose is to make the
construction operational for readers familiar with black-hole perturbation theory and
numerical-relativity (NR) waveform extraction, without narrowing the general framework. The
main text requires only that $\Rop$ vanish in the asymptotically stationary Kerr limit and
that the induced closure energy be monotone under relaxation.

For example, in a radiative-field realization, $\mathcal{Y}$ may be taken to consist of
Newman--Penrose scalars $\psi_4$ evaluated in asymptotic coordinates minus their best
Kerr-consistent projections onto spin-weighted spheroidal harmonic modes at fixed complex
frequency.

\subsection{Radiative data and Kerr-consistent projection}

Let $h(t,\theta,\phi)$ denote a strain-like observable extracted in the wave zone (or at null
infinity), and let $\psi_4(t,\theta,\phi)$ denote the Newman--Penrose curvature scalar used in
many NR pipelines. For definiteness, we take $\psi_4$ as the radiative field. In a Kerr
background, separated perturbations may be expanded in spin-weighted spheroidal harmonics
${}_{-2}S_{\ell m}(\theta,\phi;\,a\omega)$ with complex frequencies $\omega$ determined by QNM
boundary conditions. The key point for the present construction is that the angular basis is
\emph{frequency-locked} through the combination $a\omega$, so that ``angular classification''
and spectral content must be treated self-consistently.

Fix a target sector $(\ell,m)$ and a candidate complex frequency $\omega$ (e.g.\ a QNM
frequency $\omega_{\ell m n}$). Define the Kerr-consistent angular projection operator
$\Pi_{\ell m}(\omega)$ acting on functions on the sphere by
\begin{equation}
  \begin{aligned}
    \Pi_{\ell m}(\omega)\, f(\theta,\phi)
    &\;\equiv\;
    \left\langle {}_{-2}S_{\ell m}(\cdot;\,a\omega),\, f(\cdot)\right\rangle_{\Omega}\;
    {}_{-2}S_{\ell m}(\theta,\phi;\,a\omega),
  \end{aligned}
  \label{eq:Pi_lm_omega}
\end{equation}
where $\langle\cdot,\cdot\rangle_{\Omega}$ is the standard $L^2$ inner product on the sphere
(with the appropriate spin-weight measure). In practice, $\Pi_{\ell m}(\omega)$ may be
implemented by numerical evaluation of ${}_{-2}S_{\ell m}$ and quadrature on the extraction
sphere, or by mapping through a spherical-harmonic basis using known spheroidal--spherical
mixing coefficients.

We emphasize that \cref{eq:Pi_lm_omega} is \emph{not} assumed to commute with the full evolution
operator at early times; it is an \emph{adiabatic classifier} that becomes accurate as the
spacetime approaches Kerr. The closure residual below measures precisely the mismatch between
a radiative representation and its Kerr-consistent projection.

\subsection{A closure residual as projection mismatch}

Let $\mathcal{S}$ denote a space of radiative states (for example, the space of time series
with values in $L^2(S^2)$ over a chosen analysis window). Define the residual operator
$\Rop:\mathcal{S}\to\mathcal{Y}$ by
\begin{equation}
  \begin{aligned}
  \Rop[\psi_4](t,\theta,\phi)
  \;\equiv\;
  \psi_4(t,\theta,\phi) - \sum_{\ell,m}\Pi_{\ell m}(\omega_{\ell m 0})\,\psi_4(t,\theta,\phi),
  \end{aligned}
  \label{eq:Rop_projection_mismatch}
\end{equation}
where $\omega_{\ell m 0}$ may be taken as the fundamental Kerr QNM frequency in each sector
(using a remnant mass and spin inferred from the inspiral or NR). The sum in
\cref{eq:Rop_projection_mismatch} may be restricted to the dominant sectors used in the
analysis; its precise truncation is not essential for the formal structure.

Interpretation: $\Rop[\psi_4]$ is the component of the radiative field that fails to be
captured by the Kerr-consistent angular decomposition at the reference frequencies. In the
late-time Kerr limit, where $\psi_4$ is well-described by separated QNMs in spheroidal
harmonics, this mismatch tends to zero (up to numerical truncation and finite-radius effects).

\paragraph*{Linearization.}
Let $X_\star$ denote the asymptotically stationary Kerr state, and let $x$ denote a
perturbation in the radiative tangent space $\Xrad$. Linearizing
\cref{eq:Rop_projection_mismatch} about $X_\star$ yields
\begin{equation}
  D\Rop_{X_\star}x
  \;\approx\;
  x - \sum_{\ell,m}\Pi_{\ell m}(\omega_{\ell m 0})\,x,
  \label{eq:DRop_linearized_projection}
\end{equation}
where we have suppressed the dependence of $\Pi_{\ell m}$ on the remnant parameters for
notational clarity. More refined linearizations may include the variation of
$\Pi_{\ell m}(\omega)$ under perturbations of $\omega$ and of the remnant parameters; such
terms are higher order in the present methods note and are naturally incorporated in
catalog-scale applications.

\subsection{Canonical weight from radiated-energy norms}

The residual space $\mathcal{Y}$ may be taken as the same function space as the extracted
radiative field (e.g.\ time series valued in $L^2(S^2)$), and the weight $W$ in
\cref{eq:closure_energy} may be chosen to coincide with a physically motivated quadratic norm.
A natural choice, consistent with the interpretation of $\Rop$ as ``non-Kerr-consistent''
radiative content, is a flux-like norm on $\psi_4$ residuals:
\begin{equation}
  \normW{y}^2
  \;\equiv\;
  \int_{t_0}^{t_1}\! dt\int_{S^2}\! d\Omega\; \alpha_W(t)\, |y(t,\theta,\phi)|^2,
  \label{eq:W_flux_norm}
\end{equation}
where $\alpha_W(t)\ge 0$ is an optional taper or whitening weight (for example, to suppress
window-edge artifacts or to incorporate detector noise weighting). With $\alpha_W\equiv 1$,
\cref{eq:W_flux_norm} is an $L^2$ norm in time and angle. More sophisticated $W$ may incorporate
frequency weighting consistent with the standard energy flux at null infinity expressed in
terms of $\psi_4$; for the legitimacy diagnostic, only positivity and physical invariance
properties are required.

Under \cref{eq:DRop_linearized_projection} and \cref{eq:W_flux_norm}, the closure energy
\cref{eq:closure_energy} becomes
\begin{equation}
  \begin{aligned}
    r(t,\theta,\phi)
    &\equiv
    x(t,\theta,\phi)
    - \sum_{\ell,m}\Pi_{\ell m}(\omega_{\ell m 0})\,x(t,\theta,\phi),
    \\
    \Eop(x)
    &= \tfrac12 \int_{t_0}^{t_1}\! dt\int_{S^2}\! d\Omega\;\alpha_W(t)\,|r(t,\theta,\phi)|^2,
  \end{aligned}
  \label{eq:Eop_projection_energy}
\end{equation}
which penalizes radiative content that fails to admit a Kerr-consistent angular
representation on the analysis window. This provides an explicit example of a canonical
weight class: $W$ is fixed up to overall scale by demanding invariance under admissible
reparameterizations of the residual field (e.g.\ changes of angular basis and normalization)
and by selecting a flux-like norm corresponding to physically radiated energy in the chosen
representation.

\subsection{Remarks on scope}

The construction above is intended as a concrete realization for the ringdown application,
not as a unique definition. Alternative realizations of $\Rop$ are equally compatible with the
formalism, including (i) constraint-residual constructions on Cauchy slices using Hamiltonian
and momentum constraints, (ii) gauge-consistency residuals, and (iii) residuals defined by
mismatch between NR-extracted $\psi_4$ and solutions of the Teukolsky equation driven by a
best-fit Kerr background. In all cases, the legitimacy criterion and pseudospectral isolation
diagnostics depend only on the induced positive semidefinite closure-depth operator
$\Dop=(D\Rop_{X_\star})^\ast W D\Rop_{X_\star}$ and on the non-normal evolution structure.

\paragraph*{Toy non-normal analogy.}
To further illustrate the closure-residual construction independently of general-relativistic
details, consider a simple finite-dimensional analogy. Let $L$ be a non-normal $2\times2$
linear operator with eigenvalues having negative real parts but non-orthogonal eigenvectors,
so that transient amplification occurs despite asymptotic decay. A signal evolved under $L$
may be accurately reconstructed at early times by including multiple decaying exponentials,
even though these components do not correspond to independent relaxation channels. In this
setting, the closure residual corresponds to the projection mismatch between a truncated modal
reconstruction and the full state, while the closure energy is simply the squared norm of the
discarded component. Early-time inclusion of additional modes can reduce pointwise residuals
while increasing the closure energy, reflecting non-normal mixing rather than genuine decay
channels. Only once the dominant eigenmode becomes spectrally isolated does the closure energy
decay monotonically, recovering a physically interpretable modal description. This toy example
captures, in minimal form, the same distinction between fitting accuracy and dynamical
legitimacy emphasized in the Kerr ringdown context.

