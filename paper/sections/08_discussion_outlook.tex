\section{Discussion and outlook}
\label{sec:discussion}

The proposed diagnostics shift ringdown interpretation from ``how many damped sinusoids fit''
to ``which components represent genuine relaxation channels of a non-normal operator.''
This reframing provides a principled explanation for start-time ambiguity and early-time mode
proliferation without requiring modifications to General Relativity or Kerr QNM theory.

Immediate next steps include applying the diagnostics to publicly available numerical-relativity
catalogs to quantify typical isolation times $t_\star$ as functions of mass ratio, spin,
eccentricity, and precession. A second direction is end-to-end integration with inference
pipelines, using closure-energy monotonicity as a structural prior or model-acceptance gate.

The framework is NR-ready in the sense that all required ingredients are already present in
current waveform extraction practice: Appendix~\ref{app:closure_residual_example} gives an
explicit closure residual in terms of $\psi_4$ and Kerr-consistent projections, and reduced
resolvent estimates may be constructed using standard QNM solvers and angular projectors. A
catalog-scale evaluation of $t_\star$ and legitimacy classification on public SXS/RIT
waveforms is deferred to future work.

As a concrete validation target, applying the closure-residual and pseudospectral diagnostics
to individual public NR waveforms (e.g.\ selected SXS or RIT simulations) will enable direct
comparison between isolation times, legitimacy classification, and conventional overtone
fits in fully nonlinear merger data.

While specific thresholds (e.g.\ $\eta$ and time-smoothing scales) are required operationally,
the diagnostics are designed to be qualitative and order-of-magnitude stable, with conclusions
insensitive to reasonable variations of these parameters.

\paragraph*{Testable prediction.}
Because pseudospectral isolation is controlled by non-normal mixing strength, this framework
predicts delayed ringdown onset in systems with stronger mode coupling. In particular, mergers
with high mass ratio, significant precession, or near-extremal remnant spin are expected to
exhibit larger inferred isolation times $t_\star$ than comparable low-mixing systems, even
when waveform mismatch appears stable at earlier times. This prediction can be tested
directly against public numerical-relativity catalogs by correlating inferred $t_\star$ with
source parameters across families of simulations.

In high-SNR events, statistical uncertainties can become smaller than systematic modeling
errors \cite{Isi2019,LIGORingdown2021}. Under these conditions, pseudospectral artifacts can masquerade as additional modes or
as deviations from Kerr predictions. The legitimacy and isolation diagnostics provide an
operator-level firewall against such false positives by enforcing a physically motivated
interpretation criterion for mode claims.
