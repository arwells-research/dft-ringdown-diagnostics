\section{Pseudospectral isolation and ringdown onset}
\label{sec:pseudospectral_isolation}

Non-normality implies that spectral separation alone is insufficient to guarantee stable
modal interpretation. In such systems, stability and interpretability are governed by the
resolvent norm $\|(zI-\Lop)^{-1}\|$, rather than by eigenvalues alone.

\subsection{Pseudospectrum}

For $\varepsilon>0$, define the $\varepsilon$-pseudospectrum \cite{Trefethen1997,TrefethenEmbree2005}
\begin{equation}
  \pspec(\Lop) \equiv \left\{ z\in\mathbb{C}\,:\, \|(zI-\Lop)^{-1}\| > \varepsilon^{-1}\right\}.
  \label{eq:pseudospec}
\end{equation}
When $\Lop$ is non-normal, $\pspec(\Lop)$ can extend far beyond neighborhoods of the spectrum
$\spec(\Lop)$, producing transient amplification, strong mode interference, and spectral
instability under small perturbations. Such behavior has been shown to be relevant for black
hole ringdown and waveform modeling, particularly at early times following merger
\cite{Boyd2022}.

In practice, $\varepsilon$ need not be specified independently; pseudospectral structure is
probed operationally through resolvent norm ratios near candidate modal frequencies, as
described below.

\subsection{Reduced-sector resolvent diagnostics}

Direct computation of $\pspec(\Lop)$ is typically intractable for high-dimensional operators.
We therefore restrict attention to approximately invariant angular sectors. Let $P_{\ell m}$
denote a projector onto an approximate $(\ell,m)$ subspace, defined for example by symmetry
considerations and adiabatic angular projectors appropriate to Kerr perturbation theory.
Define the reduced operator
\begin{equation}
  \Lop_{\ell m} \equiv P_{\ell m}\,\Lop\,P_{\ell m}.
\end{equation}
For candidate eigenvalues $\lambda_{\ell m n}$ in this sector, define local resolvent
operator magnitudes
\begin{equation}
  R_{\ell m}(z) \equiv \left\|(zI - \Lop_{\ell m})^{-1}\right\|.
  \label{eq:Rlm}
\end{equation}

The reduced-sector approximation is justified when inter-sector couplings are perturbative on
the time window of interest, i.e.,
\(
\|P_{\ell m}\Lop P_{\ell' m'}\| \ll \|P_{\ell m}\Lop P_{\ell m}\|
\)
for $(\ell',m')\neq(\ell,m)$ in an appropriate operator norm. In systems with strong symmetry
breaking---such as highly precessing, eccentric, or near-extremal-spin remnants---this
condition may fail. In such cases, the diagnostic can be applied to an expanded block
containing the dominant coupled sectors rather than a single $(\ell,m)$ subspace. The
framework itself does not require strict sector decoupling; it requires only that the chosen
reduced operator capture the dominant mixing channels relevant to the analysis window.

\subsection{Operational definition of ringdown start time}

We define the ringdown onset time $t_\star$ as the earliest time at which the fundamental mode
in a target sector becomes resolvent-isolated relative to nearby candidate modes:
\begin{equation}
  t_\star \equiv \inf\left\{t:\;
  \frac{R_{\ell m}(\lambda_{\ell m 0})}{\max_{k\neq 0} R_{\ell m}(\lambda_{\ell m k})} < \eta
  \right\},
  \label{eq:tstar}
\end{equation}
where $\eta\ll 1$ is a robustness threshold.

Operationally, $\eta$ sets the tolerance for resolvent isolation rather than defining a sharp
spectral boundary. In synthetic tests we find that qualitative conclusions are stable under
order-of-magnitude variations in $\eta$, provided it is chosen well below unity; throughout
this work we use $\eta \sim 0.1$ as a conservative reference value.

In particular, we observe stable qualitative behavior over the range
$0.03 \lesssim \eta \lesssim 0.3$ in the synthetic experiments reported below; values closer
to unity blur isolation by construction, while excessively small values primarily delay
reported $t_\star$ without changing the monotonicity classification.

At times earlier than the onset time $t_\star$, pseudospectral overlap prevents distinct modal components from being
physically separable: multiple basis elements may be required to reconstruct the waveform,
but their coefficients cannot be interpreted as independent relaxation amplitudes. For
$t\ge t_\star$, dominant modes become resolvent-isolated, and modal interpretations stabilize
under window shifts and basis refinements, providing an operator-based criterion for the onset
of physically interpretable ringdown.

\paragraph*{Relation to existing approaches.}
Conventional ringdown analyses typically determine start times using waveform mismatch
stabilization or Bayesian model selection, while overtone-based studies interpret early-time
improvements in fit quality as evidence for additional quasinormal modes
\cite{Giesler2019,Isi2019}. Subsequent work has emphasized ambiguities associated with nonlinear
merger dynamics, mode mixing, and basis dependence at early times
\cite{Ota2021,Boyd2022}. The present framework is complementary: rather than asking whether a
given basis element improves a fit, it asks whether its inclusion corresponds to a
resolvent-isolated relaxation channel that monotonically reduces projection inconsistency. In
this sense, the legitimacy criterion and pseudospectral isolation test reframe ongoing debates
about overtone interpretation in terms of non-normal dynamics, without introducing new physics
beyond linearized General Relativity.
