\section{Relation to existing ringdown diagnostics}
\label{sec:relation_prior}

\subsection{Mismatch-based start time selection}

Conventional pipelines often select ringdown start time by minimizing waveform mismatch or by
shifting the fit window until inferred parameters stabilize \cite{London2014,LIGORingdown2021}. The pseudospectral onset time
$t_\star$ defined in \cref{eq:tstar} provides an operator-based explanation for why such
heuristics succeed only beyond a certain time: before isolation, pseudospectral non-normal mixing makes
modal parameters unstable and basis-dependent.

\subsection{Overtone analyses}

Overtone-based reconstructions can improve early-time fits \cite{Giesler2019,Isi2019}. The present framework does not
dispute this empirical observation; instead, it explains early-time overtone proliferation as
a consequence of pseudospectral overlap. The legitimacy criterion \cref{eq:legitimacy}
distinguishes overtones that act as genuine relaxation channels from those that merely patch
prompt transients or non-modal dynamics prior to spectral isolation.

\subsection{Bayesian mode selection}

Bayes-factor approaches quantify statistical preference among candidate modal models
\cite{Isi2019,LIGORingdown2021}, but do
not, by themselves, distinguish fitting functions from physical relaxation channels in
non-normal systems. The closure-energy legitimacy criterion is complementary: it provides a
theory-driven filter that can be applied alongside Bayesian inference, particularly in the
high-SNR regime where systematic errors can dominate.
