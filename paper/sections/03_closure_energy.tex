\section{Closure residual, canonical weight, and legitimacy}
\label{sec:closure_legitimacy}

\subsection{Closure residual and closure energy}

We introduce a closure residual operator
\begin{equation}
  \Rop: \mathcal{S} \rightarrow \mathcal{Y},
\end{equation}
intended to quantify projection inconsistency between complementary representations of the
same underlying physical state. 

\emph{Interpretation of the target space.}
The space $\mathcal{Y}$ represents the linearized space of \emph{projection inconsistency
observables}: quantities that vanish when the post-merger spacetime admits a consistent
simultaneous description in both a local radiative representation and an asymptotically
stationary (Kerr-like) phase representation. In practice, elements of $\mathcal{Y}$ may be
realized through residuals of constraint-satisfying fields, curvature scalars, or radiative
fluxes evaluated in an asymptotic basis; the present framework does not require committing to
a specific realization.

Linearizing around $X_\star$ yields
\begin{equation}
  D\Rop_{X_\star}: \Xrad \rightarrow \mathcal{Y}.
\end{equation}

\emph{Physical interpretation and example construction.}
In the present application, $\Rop$ measures the degree to which a perturbed post-merger
spacetime fails to admit a mutually consistent description in two complementary
representations: (i) a local radiative description appropriate to waveform extraction, and
(ii) an asymptotically stationary description appropriate to Kerr ringdown. In the language
of General Relativity, this inconsistency may be viewed as residual violation of vacuum
Einstein constraints or gauge-consistency conditions when expressed in a basis adapted to the
late-time Kerr geometry, decaying as the remnant relaxes.

Equivalently, $\Rop$ measures the failure of two nominally equivalent projections of the same
physical state---one adapted to outgoing radiation and one adapted to stationarity---to
agree when linearized about the final Kerr background. Closure corresponds to the vanishing of
this projection mismatch.

Define the quadratic closure energy
\begin{equation}
  \Eop(x) \equiv \tfrac12 \normW{D\Rop_{X_\star} x}^2,
  \label{eq:closure_energy}
\end{equation}
where $W$ is a positive weight (inner product) on $\mathcal{Y}$.
The role of $W$ is to specify which physical ``amount'' of inconsistency is penalized.

\emph{Canonical choice of $W$.}
Rather than treating $W$ as a freely tunable object, we assume it is fixed (up to an overall
scale) by two requirements: (i) invariance under admissible reparameterizations of the
residual variables (e.g.\ changes of gauge, normalization, or basis that preserve the
physical meaning of constraint violation), and (ii) monotonic decay under physical relaxation.
Operationally, this corresponds to choosing $W$ such that $\Eop$ coincides, to leading order,
with the physically radiated energy associated with eliminating constraint or
projection-inconsistency residuals.

In the present context, admissible reparameterizations include changes of gauge, slicing, and
representation that preserve the physical radiative content of the spacetime. The requirement
of monotonic decay under relaxation fixes $W$ up to an overall normalization and excludes
weights that would artificially amplify non-radiative or gauge-dependent components.

In practice, this identifies $W$ with the quadratic form induced by the energy norm governing
constraint damping or curvature relaxation in the chosen representation. While $W$ may not be
strictly unique, physically reasonable choices related by residual-space reparameterizations
yield equivalent monotonicity behavior for $\Eop$. We treat the existence of such a canonical
class of weights as a technical assumption in this methods note; sensitivity to alternative
physically motivated choices is examined in the synthetic demonstrations below and deferred
for systematic study in future catalog-scale analyses.

Define the closure-depth operator on $\Xrad$:
\begin{equation}
  \Dop \equiv (D\Rop_{X_\star})^\ast\, W\, D\Rop_{X_\star},
  \qquad
  \Eop(x) = \tfrac12\,\inner{x}{\Dop x}.
  \label{eq:Dop_def}
\end{equation}
By construction, $\Dop$ is positive semidefinite.

\subsection{Legitimacy criterion}

Let $\hat{x}(t)$ be a truncated reconstruction as in \cref{eq:trunc_recon}.
We define:

\begin{quote}
\textbf{Legitimacy criterion.}
A candidate mode family (or truncated reconstruction) is $\Sigma_2$-legitimate on a time
interval $[t_0,t_1]$ if the reconstructed closure energy is non-increasing,
\begin{equation}
  \frac{d}{dt}\Eop(\hat{x}(t)) \le 0,
  \qquad t\in[t_0,t_1],
  \label{eq:legitimacy}
\end{equation}
to leading order.
\end{quote}

The criterion is deliberately comparative: if adding a mode improves waveform residuals but
induces growth or sustained oscillations in $\Eop(\hat{x}(t))$ relative to a simpler
reconstruction, that mode is interpreted as a non-modal reconstruction element used to patch
early-time non-normal mixing rather than as an independent physical relaxation channel.
