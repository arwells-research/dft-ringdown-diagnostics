\section{Motivation}

Black hole spectroscopy aims to test strong-field General Relativity by comparing observed
ringdown frequencies and damping rates to Kerr quasinormal mode predictions. In practice,
high-SNR events highlight two persistent difficulties: (i) the inferred mode content depends
sensitively on the chosen ringdown start time, and (ii) early-time fits can be improved by
including additional overtones, raising concerns about overfitting prompt transients or
model mismatch.

These issues are frequently treated as signal-processing or window-selection problems
\cite{Giesler2019,London2014,Ota2021}.
Here we reframe them as an operator-physics problem: dissipative ringdown dynamics are
generally governed by a \emph{non-normal} linearized evolution operator, for which eigenvalue
stability and mode separability are controlled by the pseudospectrum rather than the spectrum
alone \cite{Trefethen1997,TrefethenEmbree2005,Boyd2022}.

We further propose a mode-legitimacy criterion based on monotonic reduction of a
closure-energy functional. Informally, a genuine relaxation mode should reduce an
appropriately defined measure of projection inconsistency (a ``closure residual'') as the
post-merger spacetime relaxes toward an asymptotically stationary state. This yields a
principled distinction between basis elements that merely improve waveform fits and modes
that correspond to independent physical relaxation channels.
