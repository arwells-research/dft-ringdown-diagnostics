\section{Linearized ringdown as a non-normal relaxation problem}
\label{sec:operator_model}

Let $X_\star$ denote an asymptotically stationary endpoint of the post-merger evolution, and
let $\Xrad$ be a radiative tangent space about $X_\star$ whose elements contribute to the
observed gravitational waveform. We model ringdown as the linearized evolution
\begin{equation}
  \dot{x}(t) = \Lop\,x(t), \qquad x(t)\in \Xrad,
  \label{eq:lin_evol}
\end{equation}
with $\Lop$ a closed linear operator. In black hole perturbation theory, the spectrum of $\Lop$ corresponds to the Kerr quasinormal
mode frequencies \cite{Leaver1985,Berti2009}.

In dissipative systems, $\Lop$ is generally \emph{non-normal}, i.e.\ $[\Lop,\Lop^\ast]\neq 0$.
Consequently, eigenvectors need not form an orthogonal basis and transient growth or strong
mode interference may occur even when all eigenvalues are damped. A natural modal framework
is therefore biorthogonal \cite{TrefethenEmbree2005}: right eigenvectors $v_j$ and left
eigenvectors $w_j$ satisfy
\begin{equation}
  \Lop v_j = \lambda_j v_j, \qquad \Lop^\ast w_j = \overline{\lambda_j}\, w_j,
\end{equation}
with normalization $\inner{w_i}{v_j}=\delta_{ij}$ when possible (e.g.\ for a selected set of
isolated modes).

A \emph{truncated modal reconstruction} is defined by restricting to a finite index set
$J$:
\begin{equation}
  \hat{x}(t) \equiv \sum_{j\in J} c_j\, e^{\lambda_j t} v_j, \qquad
  c_j \equiv \inner{w_j}{x(0)}.
  \label{eq:trunc_recon}
\end{equation}
This formulation makes explicit that measured modal amplitudes depend on left-eigenvector
projections and can therefore be sensitive to non-normal conditioning at early times.

Eigenvalues alone do not determine stability. Instead, stability is governed by the
\emph{resolvent} $(zI-\Lop)^{-1}$ and by the associated pseudospectrum, reviewed in
\cref{sec:pseudospectral_isolation}.
