We introduce an operator-theoretic diagnostic for assessing the physical legitimacy of
quasinormal-mode (QNM) claims in black hole ringdown analyses. The framework treats ringdown
as a generally non-normal relaxation process and leverages biorthogonal expansions and
pseudospectral stability to define an objective onset time for spectrally interpretable
ringdown, particularly in high–signal-to-noise ratio events. A $\Sigma_2$-motivated legitimacy criterion is proposed: candidate modal
reconstructions should monotonically reduce a closure-energy functional associated with
projection inconsistency. We demonstrate, on synthetic numerical-relativity--like ringdown
data, that additional overtones can improve early-time waveform fits while failing the
legitimacy criterion, consistent with their role as pseudospectral reconstruction elements
prior to spectral isolation. The proposed diagnostics complement existing mismatch- and
Bayes-factor--based pipelines by providing an operator-physics criterion that distinguishes
fitting functions from physical relaxation channels in high-SNR events.
